%%%%%%%%%%%%%%%%%%%%%%%%%%%%%%%%%%%%%%%%%
% Structured General Purpose Assignment
% LaTeX Template
%
% This template has been downloaded from:
% http://www.latextemplates.com
%
% Original author:
% Ted Pavlic (http://www.tedpavlic.com)
%
% Note:
% The \lipsum[#] commands throughout this template generate dummy text
% to fill the template out. These commands should all be removed when 
% writing assignment content.
%
%%%%%%%%%%%%%%%%%%%%%%%%%%%%%%%%%%%%%%%%%

%----------------------------------------------------------------------------------------
%	PACKAGES AND OTHER DOCUMENT CONFIGURATIONS
%----------------------------------------------------------------------------------------

\documentclass{article}

%\usepackage[brazilian]{babel}
%\usepackage[german]{babel}
\usepackage[utf8]{inputenc}
\usepackage{fancyhdr} % Required for custom headers
\usepackage{lastpage} % Required to determine the last page for the footer
\usepackage{extramarks} % Required for headers and footers
\usepackage{graphicx} % Required to insert images
\usepackage{float}
\usepackage{listings}
\usepackage{alltt}
\usepackage{hyperref}

\graphicspath{ {img/} }

% Margins
\topmargin=-0.45in
\evensidemargin=0in
\oddsidemargin=0in
\textwidth=6.5in
\textheight=9.0in
\headsep=0.25in 

% Font
\renewcommand*{\familydefault}{\sfdefault}

% Set up the header and footer
\pagestyle{fancy}
\lhead{\hmwkAuthorName} % Top left header
%\chead{\hmwkClass\ (\hmwkClassInstructor\ \hmwkClassTime): \hmwkTitle} % Top center header
\rhead{\hmwkClass: \hmwkTitle} % Top center header
%\rhead{\firstxmark} % Top right header
\lfoot{\lastxmark} % Bottom left footer
\cfoot{} % Bottom center footer
\rfoot{Page \thepage\ of\ \pageref{LastPage}} % Bottom right footer
\renewcommand\headrulewidth{0.4pt} % Size of the header rule
\renewcommand\footrulewidth{0.4pt} % Size of the footer rule

%\setlength\parindent{0pt} % Removes all indentation from paragraphs

%----------------------------------------------------------------------------------------
%	DOCUMENT STRUCTURE COMMANDS
%	Skip this unless you know what you're doing
%----------------------------------------------------------------------------------------

% Header and footer for when a page split occurs within a problem environment
\newcommand{\enterProblemHeader}[1]{
\nobreak\extramarks{#1}{}\nobreak
\nobreak\extramarks{#1 }{}\nobreak
}

% Header and footer for when a page split occurs between problem environments
\newcommand{\exitProblemHeader}[1]{
\nobreak\extramarks{#1}{}\nobreak
\nobreak\extramarks{#1}{}\nobreak
}

\setcounter{secnumdepth}{0} % Removes default section numbers
\newcounter{homeworkProblemCounter} % Creates a counter to keep track of the number of problems

\newcommand{\homeworkProblemName}{}
\newenvironment{homeworkProblem}[1][Task \arabic{homeworkProblemCounter}]{ % Makes a new environment called homeworkProblem which takes 1 argument (custom name) but the default is "Problem #"
\stepcounter{homeworkProblemCounter} % Increase counter for number of problems
\renewcommand{\homeworkProblemName}{#1} % Assign \homeworkProblemName the name of the problem
\section{\homeworkProblemName} % Make a section in the document with the custom problem count
\enterProblemHeader{\homeworkProblemName} % Header and footer within the environment
}{
\exitProblemHeader{\homeworkProblemName} % Header and footer after the environment
}

\newcommand{\problemAnswer}[1]{ % Defines the problem answer command with the content as the only argument
\noindent\framebox[\columnwidth][c]{\begin{minipage}{0.98\columnwidth}#1\end{minipage}} % Makes the box around the problem answer and puts the content inside
}

\newcommand{\homeworkSectionName}{}
\newenvironment{homeworkSection}[1]{ % New environment for sections within homework problems, takes 1 argument - the name of the section
\renewcommand{\homeworkSectionName}{#1} % Assign \homeworkSectionName to the name of the section from the environment argumen
\subsection{\homeworkSectionName} % Make a subsection with the custom name of the subsection
\enterProblemHeader{\homeworkProblemName\ [\homeworkSectionName]} % Header and footer within the environment
}{
\enterProblemHeader{\homeworkProblemName} % Header and footer after the environment
}
   
%----------------------------------------------------------------------------------------
%	NAME AND CLASS SECTION
%----------------------------------------------------------------------------------------

\newcommand{\hmwkTitle}{Thesis Proposal} % Assignment title
\newcommand{\hmwkDueDate}{October 2015} % Due date
\newcommand{\hmwkClass}{Bachelorarbeit $-$ Informatik} % Course/class
\newcommand{\hmwkClassFull}{Proposal of \emph{Bachelorarbeit} $-$ B.Sc. Informatik} % Course/class
\newcommand{\hmwkClassTime}{Wintersemester 2015/2016} % Class/lecture time
\newcommand{\hmwkClassInstructor}{} % Teacher/lecturer
\newcommand{\hmwkAuthorName}{Fernando Bombardelli da Silva $-$ 364924} % Your name
\newcommand{\hmwkAuthorRemark}{Exchange student of double degree from the UFRGS (Porto Alegre / Brazil)}

%----------------------------------------------------------------------------------------
%	TITLE PAGE
%----------------------------------------------------------------------------------------

\title{
%\vspace{2in}
\Large\textmd{\textbf{\hmwkClassFull}}\\
\normalsize{\textbf{\hmwkTitle}}\\
\normalsize\vspace{0.1in}\small{\hmwkDueDate}\\
%\vspace{0.1in}
\large{\textit{\hmwkClassInstructor\ \hmwkClassTime}}
%\vspace{3in}
}

\author{\textbf{\hmwkAuthorName}}
\date{} % Insert date here if you want it to appear below your name

%----------------------------------------------------------------------------------------

\begin{document}

%\maketitle
{\centering
\Large\textmd{\textbf{\hmwkClassFull}}\\
\normalsize{\textbf{\hmwkTitle}}\\
\normalsize\vspace{0.1in}\small{\hmwkDueDate}\\
\small{\hmwkClassInstructor\ \hmwkClassTime}\\
%\vspace{0.1in}
\large\textbf{\hmwkAuthorName}\\
\normalsize{\hmwkAuthorRemark}\\
}
%----------------------------------------------------------------------------------------
%	TABLE OF CONTENTS
%----------------------------------------------------------------------------------------

%\setcounter{tocdepth}{1} % Uncomment this line if you don't want subsections listed in the ToC

%\newpage
%\tableofcontents
%\newpage

\linespread{1.5} % Line spacing

%----------------------------------------------------------------------------------------
%	Exercício 1
%----------------------------------------------------------------------------------------

% To have just one problem per page, simply put a \clearpage after each problem

{\large
\begin{homeworkProblem}[Introduction]

This document shows a bachelor thesis proposal to be researched by the student Fernando Bombardelli da Silva (\emph{Matrikelnummer} 364924) in the \emph{Technische Universität Berlin} in the course of Computer Science (\emph{Informatik}).

The proposed research subject has theoretical bases from the areas of multi-agent system, swarm intelligence and mathematical optimization. The context of the problem to be described involves themes of growing concern nowadays, such as transportation and traffic issues, like multi-modal mobility.

\begin{homeworkSection}{Problem Description}
In current urban areas, mainly in very populated cities, there is a huge number of mobility problems. These problems have been seen with much interest by the scientific community, which has been strongly contributing to the improvement of transportation and logistic networks, thus promoting advances towards a better mobility.

The here addressed problem is known as the dial-a-ride problem (DARP). Shortly, it consists of a system with a set of requests of pick up and delivery entered by customers and a fleet of vehicles. The goal is planning the route of the vehicles and the assignment of requests to them in a feasible way, since there are constraints from both the requests and the vehicles to which a solution is subject. These can be several conditions, like location and time limit for picking up and delivering or how much time each vehicle can operate. Besides, it is not only searched a feasible solution but an optimal one that minimizes the costs of the operation defined by a function.

Many difficulties are found when trying to solve the described problem, the combinatorial nature of its solution space make it hard to treat for large inputs because of the high complexity, it leads then to a special difficulty in building a scalable application.

%- Context of the problem (Traffic problems, Alternative public transportation, goal: plan the operation and reduce its costs)\\
%- Description of the problem (Specifications, requirements, objective ...)\\
%- Difficulties in solving (high algorithmic complexity, scalability, need for REAL-TIME response ...)\\
\end{homeworkSection}

\begin{homeworkSection}{Problem Approach}

The chosen approach to solve the problem in this work is considering this as the optimization of the cost function subject to the constraints. Moreover, in order to better evaluate the results, two different methods are proposed, namely,

\begin{itemize}
	\item a mathematical program executed by a solver that delivers an exact optimal solution if there is one;
	\item an implementation of the swarming metaheuristic Firefly to obtain a near-optimal solution in polynomial time.
\end{itemize}
%Approach as an optimization problem, reducing costs of operation subject to the constraints of the customers requests.
% - 2 different ways of solving:\\
%   - Mathematical programming for exact solution with GNU LPK solver\\
%   - Implementing a swarm metaheuristic to obtain a feasible near-optimal solution (Firefly)
\end{homeworkSection}

\end{homeworkProblem}

\begin{homeworkProblem}[Justification]

It is expected that the results of this work may bring relevant contributions to the handling of the presented problem, and even of other ones. By having two distinct approaches it is possible to compare results regarding important features of the problem, such as scalability, feasibility and deviation to an optimal solution. Furthermore, the application of the relatively new firefly algorithm to the problem can show how it performs in a such a solution space.

In addition to the contributions to the understanding of the behavior of swarming metaheuristics applied to optimization problems of transportation, the new procedure of solving the problem serves as a prototype and brings a new perspective to commercial applications which seek constantly to treat the problem in a more efficient and scalable way.

With regard to the current cities' mobility, there is a growing demand for an efficient alternative to the classical means of transportation. The implementation of such a system improves the possibility of movement of the population and makes it more efficient, since it allows the decrease of the number of cars that drive through the urban network everyday causing traffic jams in big cities. Moreover, it helps to solve an demanding problem in today's society where there is an increasing number of elderly or handicapped people, who have the right to mobility and need assistance to travel in the town.

Also in an economical view this research contributes to the win of new markets by companies who aim to enter the branch of public transportation since its main goal is minimizing the operation costs. With such a model a company can take great advantages against competitors in order to capture marketplace.

At last, the concerns of the proposed model shows also an ecological relevance by enabling the decrease of the emission of greenhouse gases in the urban area, directly, considering that in this case costs are direct proportional to the consume of fuel, and consequently to the emission of gases, such as CO$_{2}$, and indirectly by reducing the circulation of other automobiles.

%- Expected results and contribution (Evaluation of the application of swarm metaheuristic to solve the transportation problem comparing %to an exact solution, feasibility, scalability, optimality ...)\\
%- Importance for today's mobility in urban areas with alternative public transportation systems\\
%- Economical relevance, costs minimization, concurrence advantages\\
%- Ecological relevance, waste of energy and carbon fuels

\end{homeworkProblem}

\begin{homeworkProblem}[Literature Review]

The Dial-a-ride Problem (DARP) is very similar to other problems studied in the scientific society, namely, it can be cited the Pickup and Delivery Vehicle Routing Problem (PDVRP) and the Vehicle Routing Problem with Time Windows (VRPTW), problems with application in logistics. according to \cite{ref:cordeau2007}, what basically differs the DARP from these other ones is the human perspective, by the fact that people are transported. It often appears presenting two goals, minimizing operation costs subject to the constraints and maximizing the availability and quality of the service. The quality criteria include frequently aspects like route duration, customer waiting and ride time, maximum vehicle ride time, among others and are usually treated in the constraints of the optimization.

\cite{ref:cordeau2007} realized a survey on the subject, he presents three mathematical models that occur in the literature, two as formulation of a mathematical optimization and one as a scheduling problem. Therefore there are several other approaches made by other authors.

Although there are works handling the problem in an exact way, that tries to find an optimal solution, for instance, with the branch-and-cut algorithm, most of them focus on applying a determined heuristic in order to find a near-optimal solution in the search space, for example, tabu search, genetic algorithms or simulated annealing.

What has not been tried is using swarm-based metaheuristics to solve the problem, such as particle swarming optimization, ant colony or the firefly algorithm. Swarm intelligence is a technique applied in the computer science, more precisely in artificial intelligence and operations research, that is based on nature patterns or behaviors. In this point of view, these metaheuristics resemble the genetic algorithms, since these are also nature-based, but they differ by the fact that, genetic algorithms have mutation and crossover operators and are based on the theory of evolution of the species, whereas swarm intelligence techniques are based on the observation of the behavior of swarms.

In this work we will study and apply the firefly algorithm (FA), that has been showing good results in the solution of nonlinear global optimization problems. \cite{Yang2012} presents a theoretical analysis on swarm intelligence having as study cases the firefly algorithm and particle swarm optimization (PSO). \cite{Yang2013} introduce the FA approaching parameter settings, complexity and applications in a didactic way with examples, at the end he draws a conclusion showing a growing application of the method in scientific articles and foreseeing an expansion in the subject and the improvement of such metaheuristic. Additionally, \cite{Yang2009} compares the FA against the PSO running simulations in a variety of objective function and concludes affirming the superiority of the FA over the PSO and that it is potentially more powerful in solving NP-Hard problems.

%- Short overview at the state-of-the-art regarding the problem
%- What leaks? (for instance, complete framework for the problem, or, use of swarm optimization)

\end{homeworkProblem}

\begin{homeworkProblem}[Theoretical Framework]

- Theoretical background and base\\
  - Mathematical optimization\\
  - Complexity theory\\
  - Meta heuristics\\
  - Swarm intelligence\\
  - Dial-a-ride Problem\\
  - Vehicle routing problem\\
- Optimization problem representation (Formulas)\\
- Firefly algorithm (pseudo code)\\

\end{homeworkProblem}

\begin{homeworkProblem}[Methodology]

- Prototypes implementation (WITH TRANSFER??)\\
- Description of the problem in the form of a Linear Program\\
- Solving with the GNU LPK solver tool\\
- Implement the Firefly metaheuristic in a program\\
- Evaluate the results by comparing solver against metaheuristic (against other works? Maybe)\\

\end{homeworkProblem}

\begin{homeworkProblem}[Time Schedule]

\begin{tabular}{ l | p{0.25cm} | p{0.25cm} | p{0.25cm} | p{0.25cm} | p{0.25cm} | p{0.25cm} | p{0.25cm} | p{0.25cm} | p{0.25cm} | p{0.25cm} | p{0.25cm} | p{0.25cm} | p{0.25cm} | p{0.25cm} | p{0.25cm} | p{0.25cm} | p{0.25cm} | p{0.25cm} | p{0.25cm} | p{0.25cm} |}
\textbf{Month} & \multicolumn{4}{c}{1}\vline & \multicolumn{4}{c}{2}\vline & \multicolumn{4}{c}{3}\vline & \multicolumn{4}{c}{4}\vline & \multicolumn{4}{c}{5} \vline \\ \hline
\textbf{Week}	& 1	& 2	& 3	& 4	& 5	& 6	& 7	& 8	& 9	& 10	& 11	& 12	& 13	& 14	& 15	& 16	& 17	& 18	& 19	& 20 \\ \hline
Define the topic & $\bullet$	& $\bullet$	& $\bullet$	& $\bullet$	&	&	&	&	&	&	&	&	&	&	&	&	&	&	&	& \\
Literature study &	&	& $\bullet$	& $\bullet$	& $\bullet$	& $\bullet$	& $\bullet$	& $\bullet$	&	&	&	&	&	&	& $\bullet$	& $\bullet$	&	&	&	& \\
Problem definition &	&	&	&	& 	& $\bullet$	& $\bullet$	& $\bullet$	& $\bullet$	&	&	&	&	&	& $\bullet$	&	&	&	&	& \\
Collect data &	&	&	&	&	&	&	& $\bullet$	& $\bullet$	& $\bullet$	&	&	&	&	&	&	&	&	&	& \\
Implementation &	&	&	&	&	&	&	&	& $\bullet$	& $\bullet$	& $\bullet$	& $\bullet$	& $\bullet$	&	& $\bullet$	& $\bullet$	& $\bullet$	& $\bullet$	&	& \\
Experiments &	&	& $\bullet$	& $\bullet$	&	&	&	&	&	& $\bullet$	&	& $\bullet$	& $\bullet$	& $\bullet$	&	&	& $\bullet$	& $\bullet$	& $\bullet$	& \\
Evaluation &	&	&	&	&	&	&	&	&	&	&	&	& $\bullet$	& $\bullet$	&	&	&	& $\bullet$	& $\bullet$	& $\bullet$ \\
Document writing &	&	&	&	&	&	&	& $\bullet$	& $\bullet$	&	&	&	& $\bullet$	& $\bullet$	&	&	&	& $\bullet$	& $\bullet$	& $\bullet$ \\

\end{tabular}

\end{homeworkProblem}

\begin{homeworkProblem}[Milestones]

\begin{enumerate}
\item \textbf{31st October:} Well defined subject, goals, methodology, 1st prototype of the LP
\item \textbf{30th November:} Linear program ready and executed; Problem defined; Creation of test cases; Partial documentation
\item \textbf{24th December:} Run the first version of the metaheuristic implementation
\item \textbf{24th January:} First results and evaluations done; Documentation; Adjusts in the model and in the implementation (possible)
\item \textbf{14th February:} Final evaluation of results; Documentation
\end{enumerate}

\end{homeworkProblem}

}

\begin{thebibliography}{1}

\bibitem{crescenzi}
CRESCENZI, Pierluigi; KANN, Viggo. \textbf{A compendium of NP optimization problems} (1998). Available at: \url{ftp://ftp.nada.kth.se/Theory/Viggo-Kann/compendium.pdf}

\bibitem{schrijver}
SCHRIJVER, Alexander. \textbf{A Course in Combinatorial Optimization} (2013). Available at: \url{http://homepages.cwi.nl/~lex/files/dict.pdf}

\end{thebibliography}

\end{document}
