%----------------------------------------------------------------------------------------
%	Macros para definir programa linear
%----------------------------------------------------------------------------------------
\makeatletter
\newcommand{\minproblem}{\@ifstar\minproblemstar\minproblemplain}
\newcommand{\minproblemplain}[3][]{
  \begin{align}
    \text{#1}\textbf{minimize}\qquad & #2\\
    \textbf{subject to}\qquad & #3
  \end{align}
}
\newcommand{\minproblemstar}[3][]{
  \begin{align*}
    \text{#1}\textbf{minimize}\qquad & #2\\
    \textbf{subject to}\qquad & #3
  \end{align*}
}
\newcommand{\maxproblem}{\@ifstar\maxproblemstar\maxproblemplain}
\newcommand{\maxproblemplain}[3][]{
  \begin{align}
    \text{#1}\textbf{maximize}\qquad & #2\\
    \textbf{subject to}\qquad & #3
  \end{align}
}
\newcommand{\maxproblemstar}[3][]{
  \begin{align*}
    \text{#1}\textbf{maximize}\qquad & #2\\
    \textbf{subject to}\qquad & #3
  \end{align*}
}
\makeatother